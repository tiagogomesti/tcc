\documentclass[monografia]{subfiles}

\begin{document}
	\chapter{Materiais e Métodos}

	Em Figura~\ref{fig:brasao} na página~\pageref{fig:brasao} temos um 
	exemplo de inserção de uma ilustração com título e fonte.
	% e esse é um exemplo de referência para uma ilustração e sua página

	\begin{figure}[htb]
		\centering
		\caption{O brasão da UFC}
		\includegraphics[width=4cm]{ufc.png}
		\label{fig:brasao}
		\fonte{Elaborada pelos autores.}
		%\nota{Esta é uma nota, que diz que os dados são baseados na regressão linear.}
		%\nota[Anotações]{Uma anotação adicional, seguida de várias outras.}
	\end{figure}


	\begin{table}[htb] 
		\IBGEtab
		{
			\caption{Um Exemplo de tabela alinhada que pode ser longa ou curta, conforme padrão IBGE.}
			\label{tab:ibge}
		}
		{ 
			\begin{tabular}{ccc} \toprule 
				Nome & Nascimento & Documento \\ \midrule \midrule 
				Maria da Silva & 11/11/1111 & 111.111.111-11 \\ \bottomrule 
			\end{tabular}
		}
		{
			\fonte{Produzido pelos autores.}
			%\nota{Esta é uma nota, que diz que os dados são baseados na regressão linear.}
			%\nota[Anotações]{Uma anotação adicional, seguida de várias outras.}
		} 
	\end{table}

	Em Tabela~\ref{tab:ibge} temos um exemplo de tabela seguindo o padrão do IBGE.
\end{document}
