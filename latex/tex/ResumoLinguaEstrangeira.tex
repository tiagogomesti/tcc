\documentclass[monografia]{subfiles}

\begin{document}
	% Inicia-se em folha/página distinta com a palavra ABSTRACT, 
	% RESUMEM ou RESUMÉ, conforme a língua,	
	\begin{resumo}[Abstract]
	\begin{otherlanguage*}{english}
	\noindent
	With technology taking great strides in recent decades, the electronic devices refresh rate 
	also increases, causing some devices having their manufacture discontinued.
	This work has as one of its purposes to avoid
	a new redesign of monotonic signaling processing functions, 
	because in this work was implemented a processing core described in HDL,
	and it is portable to different FPGAs, without the need to make changes to the code.
	The core described in this paper, is a controlled module, to be inserted in a system,
	which aims to be responsible for mathematical processing of the monotonic signals 
	of system.
	In developing this work, a processing tool was applied, the Goertzel algorithm,
	which is a derivative of Discrete Fourier Transform.
	The results of the processing times have shown the that work proposed has a good performance,
	reaching a maximum throughput of 222 channels, with the initial target 8, on a $125 \mu s$ time interval, 
	denoting a high processing power for a clock value of only $80 MHz$

	\vspace{\onelineskip}
	\textbf{Keywords}: Goertzel. FPGA. DSP. VHDL.
	\end{otherlanguage*}
	\end{resumo}
	
\end{document}
