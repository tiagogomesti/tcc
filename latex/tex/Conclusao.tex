\documentclass[monografia]{subfiles}

\begin{document}

	\chapter{Conclusão}
		Neste trabalho foi desenvolvido um core de detecção de sinais telefônicos monotônicos. A plataforma alvo de desenvolvimento é um hardware
		configurável FPGA, modelo  virtex 6 fabricado pela \textit{Xilinx}. 
		O hardware proposto tem duas interfaces de controle, que são a de recebimento de comandos e de envio de mensagens, 
		e uma interface de injeção de áudio, que consiste em oito canais de recepção multiplexados no tempo.
		Internamente, está implementado um processador de sinais, com arquitetura 16 bits, responsável pelo processamento matemático dos áudios.

		Foi utilizado neste trabalho o algoritmo de \textit{Goertzel}, que é a base teórica fundamental para o processamento dos sinais. Este algoritmo foi 
		implementado em hardware, possibilitando um aumento significativo de sua velocidade, pois a maior parte das operações são efetuadas em paralelo.

		O desenvolvimento foi realizado utilizando a metodologia \textit{top-down}, assim possibilitando uma maior organização do trabalho. A sequência de
		desenvolvimento seguiu o fluxo de utilização do hardware proposto, ou seja, iniciou-se pela interface de comandos, por onde passa a informação
		do áudio que será detectado, seguido pelo desenvolvimento da interface de recebimento de áudio. Tendo os parâmetros e o áudio a ser detectado, 
		deu-se início ao desenvolvimento do detector de sinais, posteriormente sendo feito o desenvolvimento da interface de mensagens, por fim retorna o resultado
		da detecção.

		Neste trabalho foram obtidos três resultados, primeiro da ferramenta de síntese, segundo dos resultados temporais, e 
		o terceiro baseados nas mensagens de respostas. Da ferramenta de síntese 
		obtemos a quantidade de elementos lógicos e slices utilizados, não foi possível fazer uma comparação destes resultados pois não foi encontrado na
		literatura métricas de comparação. Os resultados temporais mostraram que dentro da janela de processamento, o hardware proposto consegue processar muitos 
		canais além dos oito canais propostos, com uma taxa de acerto de $100\%$, chegando esse número até 
		duzentos e vinte e dois canais. Da análise das mensagens de resposta pode ser verificado que todos os resultados
		esperados aconteceram, isso acontece pelo fato de que o processamento é realizado de maneira determinística, só havendo a possibilidade de o áudio ser 
		detectado, ou não.

		\section{Trabalhos Futuros}
			Como trabalho futuro temos três metas, embarcar o código na FPGA e iniciar a fase de testes no hardware. A segunda meta é ampliar a faixa de frequência
			em que a detecção acontece, mas para isso teremos que voltar a fase de pré-projeto, pois para essa ampliação terão que ser aumentados a largura, em bits,
			de alguns sinais internos, influenciando em desempenho, pois esta ação irá influenciar diretamente nos cálculos matemáticos. A terceira meta é injetar
			áudios com ruídos, tornando a análise mais próxima da realidade em que os áudios estão submetidos no meio de comunicação.



			
\end{document}
