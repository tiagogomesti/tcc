\documentclass[monografia]{subfiles}

\begin{document}

	\chapter{Introdução}	
		Devido ao constante crescimento da tecnologia e sua participação cada vez mais presente no cotidiano,
		os sistemas digitais de computação vêm se tornando ferramenta de grande utilidade para o desenvolvimento da sociedade. 
		Sua participação permeia uma vasta gama de atividades, como quais, medicina, agricultura, controle aéreo, etc,
		fazendo com que seus avanços causem impactos significativos nos processos aos quais fazem parte.

		Os sistemas de telecomunicações também tiram proveito desta evolução tecnológica, 
		tendo na década de 60 iniciado um ciclo de digitalização de seus sistemas, que estende-se até hoje, que tem como características
		principais a compactação do hardware e o aumento do uso de softwares, o que tem como resultado equipamentos cada vez menores e com 
		maior poder de processamento. 
		No princípio, a digitalização teve maior ênfase nos sistemas de transmissão, fazendo uso da técnica de digitalização de sinais denominada
		PCM (Pulse Code Modulation). Não demorou muito e a digitalização foi também adotada nas centrais telefônicas.
		Mesmo com os avanços tecnológicos, na telefonia ainda hoje há uma coexistência de centrais telefônicas analógicas e digitais, com um decréscimo progressivo
		das centrais analógicas e incremento das centrais digitais.

		Para a correta comunicação das centrais telefônicas, seja com outras centrais ou com usuários, são usados sinalizações acústicas, que são sinais
		audíveis na faixa de frequências $[0Hz,4000Hz]$. Alguns exemplos destas sinalizações são o tom de discagem, que consiste em um sinal contínuo de $425 Hz$,
		outro exemplo é o tom de ocupado, que também é sinal de $425 Hz$ com duração de $250ms$ e interrupção de $250ms$ sucessivamente.

		Para o desenvolvimento deste trabalho, foram necessários conhecimentos em desenvolvimento de hardware em VHDL, para isto, foram utilizados 
		\cite{pedroni} e \cite{mesquita}, além disso, para estudo da ferramenta de desenvolvimento, foram utilizados \cite{virtex6_CLB}, \cite{virtex6}
		\cite{edk} e \cite{edkConcepts}.







	\section{Motivação}
		De maneira geral, as centrais telefônicas fazem o processamentos dos áudios com uso de processadores de sinais, os quais são embarcados com firmware
		para devido processamento.
		Com as atualizações das tecnologias, alguns modelos de processadores de sinais utilizados nas centrais podem ter sua fabricação
		descontinuada, fazendo com que um novo firmware para outro modelo de DSP tenha que ser projetado, demandando investimento financeiro e de pessoal,
		tendo o fato de que o código do firmware é totalmente dependente da plataforma alvo.

		Uma alternativa para este problema está no uso de FPGAs(\textit{Field Programmable Gate Array}), pois com o código do processamento escrito em uma
		linguagem de descrição de hardware, VHDL ou Verilog por exemplo, e com o uso de uma ferramenta de síntese, este código pode ser reutilizado
		para diferente FPGAs, pois a única diferença é a síntese para a FPGA alvo, permanecendo o código inalterado. 
		\cite{Kamai} e \cite{Lennart} mostram em seus trabalhos, que FPGAs apresentam um meio eficiente para desenvolvimento e produção de sistemas digitais.


	\section{Objetivos}
		A seguir são apresentados os objetivos gerais e específicos para este trabalho.

	\subsection{Objetivos Gerais}
		O principal objetivo deste trabalho é desenvolver um core de hardware, com arquitetura de 16 bits, descrito através da linguagem VHDL, 
		para a detecção de sinais monotônicos da telefonia, tendo como plataforma alvo uma FPGA da \textit{Xilinx}.



	\subsection{Objetivos Específicos}
		O trabalho proposto tem os seguintes objetivos específicos:

			\begin{enumerate}
			\label{sec:test}
				\item Desenvolver um hardware controlado, com interface simples, para fácil inserção em um sistema;
				\item Processar oito canais de detecção;
				\item Realizar o processamento dos canais sem perder amostras de áudio.

			\end{enumerate}


	\section{Organização do trabalho}
		O trabalho está organizado em cinco capítulos, incluindo este capítulo de introdução. O segundo capítulo faz um resumo da teoria de processamento de sinais
		utilizado na detecção dos áudios, aborda temas  como a transformada de fourier, transformada discreta de fourier e algoritmo de goertzel.
		O terceiro detalha o desenvolvimento do hardware do core de detecção de tons, descrevendo todos seus blocos de
		hardware internos, assim como a comunicação entre eles e suas funcionalidades. 
		O quarto capítulo apresenta os resultados obtidos da simulação do hardware e da ferramenta de síntese. O ultimo capítulo faz a conclusão e 
		apresenta sugestão para trabalhos futuros.


\end{document}
