\documentclass[monografia]{subfiles}

\begin{document}

	\chapter{Revisão de Literatura}

	\section{Modelos de Referências para Monografias}
	Monografias incluem os seguintes documentos: livros, folhetos,
trabalhos de conclusão de curso, teses, dissertações, manual, guia,
catálogo, enciclopédia, dicionário, relatórios entre outros.

	\subsection{Monografia no todo}
	Os elementos essenciais são: autor(es), título, edição, local,
editora e data de publicação.

	\subsubsection{Livros}
	Exemplos: \cite{livro01}, \cite{livro02} e \cite{livro03}.

	\subsubsection{Bíblias}
	Exemplos: \cite{biblia01} e \cite{biblia02} (ainda
	a serem corrigidos).

	\subsubsection{Relatórios}
	Exemplo: \cite{relatorio01}.

	\subsubsection{Dicionários}
	Exemplos: \cite{dicionario01}.

	\subsubsection{Enciclopédias}
	Exemplos: \cite{enciclopedia01}.

	\subsubsection{Teses, dissertações e trabalhos de conclusão de curso}
	Os elementos essenciais são: autor, título, data de publicação,
nota do tipo de documento (tese, dissertação, trabalho de conclusão de
curso etc.), grau, vinculação acadêmica, local e data de defesa ou
apresentação, mencionada na folha de aprovação.

Quando impressos apenas no anverso indica-se a quantidade de
folhas (f.). Quando impressos no anverso e verso, indica-se o total de
páginas (p.). A indicação da quantidade de folhas ou páginas é opcional.

	Exemplos: \cite{tese01}, \cite{dissertacao01}, \cite{monografia01}
	e \cite{tcc01}.

	\subsection{Monografia em formatos eletrônicos}
	As referências devem obedecer aos padrões indicados para os
documentos monográficos no todo, acrescidas das informações relativas
à descrição física do meio eletrônico (disquete, CD-ROM, DVD, online
etc.).

Quando se trata de obras consultadas online, também são
essenciais as informações sobre o endereço eletrônico, apresentado entre
os sinais < >, precedido da expressão “Disponível em:” e a data de
acesso ao documento, precedida da expressão “Acesso em:”. Pode-se
acrescentar dados referentes a hora, minutos e segundos.

	Exemplos: \cite{monodigital01} e \cite{monodigital02}.

	\subsection{Monografia em parte}
	Os elementos essenciais são autor(es), título da parte, seguidos da
expressão “In”, e da referência completa da monografia no todo. No
final da referência, deve-se informar a paginação ou outra forma de
individualizar a parte referenciada.

	Exemplos: \cite{monoparte01} e \cite{monoparte02}.

	\subsection{Monografia em parte no formato eletrônico}
As referências devem obedecer aos padrões indicados para partes
de monografias, de acordo com 5.4.3, acrescidas das informações
relativas à descrição física do meio eletrônico. Quando se tratar de obras
consultadas online, proceder-se-á conforme 5.4.2.

	Exemplos: \cite{monodigiparte01}, \cite{monodigiparte02} e \cite{monodigiparte03}
	(corrigir?).


	\section{Modelos de Referências para Publicações Periódicas}
	Incluem os seguintes documentos: revistas, jornais, anuários entre
outros documentos publicados periodicamente.

	\subsection{Publicação periódica no todo}
	Os elementos essenciais são: título, local de publicação, editora,
datas de início e de encerramento da publicação, se houver.

	Exemplos: \cite{periodicotodo01} e \cite{periodicotodo02}.

	\subsection{Parte de publicação periódica sem título próprio}
Refere-se ao volume, fascículo, suplementos, entre outros, de um
periódico.

Os elementos essenciais incluem: título da publicação, local,
editora, numeração do ano e/ou volume, numeração do fascículo,
informações de períodos e datas de sua publicação.

	Exemplos: \cite{periodicopartentit01}, \cite{periodicopartentit02} e \cite{periodicopartentit03}.

	\subsection{Parte de publicação periódica com título próprio}
	Refere-se ao volume, fascículo, suplementos, números especiais
entre outros, de um periódico.

Os elementos essenciais incluem: título da parte, título da
publicação, local de publicação, numeração correspondente ao volume
e/ou ano, data e particularidades que identificam a parte.

	Exemplo: \cite{periodicopartetit01}.

	\subsection{Artigo e/ou matéria de revista}
	Os elementos essenciais são: autor(es), título do artigo ou matéria,
título da publicação, local de publicação, numeração correspondente ao
volume e/ou ano, fascículo ou número, paginação inicial e final,
intervalo de publicação (se houver) e data.
Pode-se abreviar os títulos de periódicos, conforme a NBR 6032,
desde que seja mantida a uniformidade em todas as referências.

	Exemplos: \cite{artigorevista01}, \cite{artigorevista02} (corrigir?), \cite{artigorevista03}
	e \cite{artigorevista04}.

	\subsection{Artigo e/ou matéria de revista em meio eletrônico}

	Exemplos: \cite{artigorevistadigital01} e \cite{artigorevistadigital02}.

	\subsection{Artigo e/ou matéria de jornal}
	Os elementos essenciais são: autor(es) (se houver), título do
artigo ou matéria, título do jornal, local de publicação, data de
publicação, seção, caderno ou parte e a paginação correspondente.
Quando não houver seção, caderno ou parte, a paginação do artigo ou
matéria precede a data.

	Exemplos: \cite{artigojornal01} e \cite{artigojornal02}.

	\subsection{Artigo e/ou matéria de jornal em meio eletrônico}

	Exemplo: \cite{artigojornaldigital01}.

	\section{Modelos de Referências para Eventos}
	Inclui o conjunto de documentos resultante de um evento (anais,
atas, relatórios, entre outros). Os eventos podem ser seminários,
congressos, conferências, ou outros.

	\subsection{Evento no todo}
	Os elementos essenciais são: nome do evento, numeração (se
houver), ano e local (cidade) de realização. Em seguida, deve-se
mencionar o titulo do documento (anais, atas, tópico temático etc.),
seguido dos dados de local de publicação, editora e data de publicação.

	Exemplos: \cite{evento01}, \cite{evento02} e \cite{evento03}.

	\subsection{Evento no todo em meio eletrônico}
	Exemplos: \cite{eventodigital01} e \cite{eventodigital02}.

	\subsection{Trabalho apresentado em evento}
	Os elementos essenciais são: autor(es), título do trabalho
apresentado, seguido da expressão “In:”, nome do evento, numeração
do evento (se houver), ano, local de realização (cidade), titulo do
documento (anais, atas, tópico temático), local, editora, data de
publicação e página inicial e final da parte referenciada.

	Exemplos: \cite{trabalhoevento01} e \cite{trabalhoevento02}.

	\subsection{Trabalho apresentado em evento em meio eletrônico}
	Exemplos: \cite{trabalhoeventodigital01} e \cite{trabalhoeventodigital02}.
	

	\section{Modelos de Referências para Patente}
	\section{Modelos de Referências para Documentos Jurídicos}
	\section{Modelos de Referências para Imagem em Movimento}
	\section{Modelos de Referências para Documentos Iconográficos}
	\section{Modelos de Referências para Documentos Cartográficos}
	\section{Modelos de Referências para Documento Sonoro}
	\section{Modelos de Referências para Partitura}
	\section{Modelos de Referências para Documento Tridimensional}
	\section{Modelos de Referências para Documentos de Acesso Exclusivo em Meio Eletrônico}
	\section{Modelos de Referências para Documentos Diversos}

\end{document}
