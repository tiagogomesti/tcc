\documentclass[monografia.tex]{subfiles}

% ** Elemento opicional **
% Descrição: Texto em que o autor faz agradecimentos dirigidos
% àqueles que contribuiram de maneira relevante à elaboração do
% trabalho.

\begin{document}
	\chapter*{Agradecimentos}
    

   Agradeço a todos os amigos que fiz nesta empreitada, em especial à Ádria, Alan Cadore, Filipe Lins, George Harinson, João Marcelo,
   Luciene Lira, Marciel, Ridley e
   Victor Fernandes, com os quais compartilhei muitos momentos bons, e alguns tensos também. 
   
   Agradecimentos aos amigos Cincinato, Fábio Cisne, Jacques Bessa, José Adriano Filho e Nícolas Araujo, com os quais tive um grande aprendizado 
   em meu primeiro projeto de mercado, no Lesc.

   Agradeço aos meus professores, fontes de conhecimento infinito, capaz de ensinar não só os conteúdos técnicos, mas verdadeiras lições de vida também.

   Um agradecimento especial ao professor Jarbas, o qual sempre acreditou em mim. Com seus ensinamentos, suas histórias e seus depoimentos,
   que nos mostra que apesar de difícil, a engenharia tem uma beleza sensível a aqueles que se empolgam com seus desafios.

   Agradeço ao Lesc, laboratório que entre idas e vindas, estou desde 2010. Onde conheci pessoas acolhedoras, divertidas e acima de tudo, ultra responsáveis 
   com os projetos. Sinto orgulho de ser um pequena parte da história deste laboratório.

   Agradeço à minha família, em especial aos meus pais, Amorim e Josiêda, e minha tia Rosa Maria por todo apoio, ensinamento, cobranças e carinhos.

   Um agradecimento mais que especial à Natalia Maria, minha companheira de todos os momentos. Mulher de índole e temperamento forte, mas com um coração
   enorme e caridoso. Obrigado por todo o apoio, paciência, cobrança e sobretudo amor. 
   



	\newpage    
    
% a) Inicia-se em folha/pagina distinta com a palavra AGRADECIMENTOS
%    na margem superior, em letras maiúsculas, em negrito sem indicativo
%    numérico, em espaço 1,5 de entrelinhas e centralizada. O texto deve
% b) O texto deve ser em espaço 1,5 entrelinhas e justificado.

\end{document}
