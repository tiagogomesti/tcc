\documentclass[monografia]{subfiles}

\begin{document}
	% anverso
	%\begin{folhaderosto}
	\begin{center}
	% a) nome do autor
		{\ABNTEXsubsubsubsectionfont\MakeTextUppercase{\imprimirautor}}
		\vspace*{\fill}
		
	% b) título do trabalho
	% c)subtítulo (se houver), separado do título por dois pontos para
	% evidenciar a subordinação ao título
		\ABNTEXsubsubsubsectionfont\MakeTextUppercase{\imprimirtitulo}
		\vspace*{\fill}
		
	% d)número do volume. Se houver mais de um, deve constar em
	% cada folha de rosto o respectivo volume em algarismos arábicos;

	% e)natureza – nota contendo o tipo do trabalho (tese, dissertação,
	% trabalhos de conclusão de curso e outros) e objetivo
	% (aprovação em disciplina, grau pretendido e outros); nome da
	% instituição a que é submetido; área de concentração;
		\begin{SingleSpacing}
		\imprimirnatureza{\imprimirtipotrabalho\\ 

	% f)nome do orientador e, se houver, do coorientador;
		\imprimirorientadorRotulo\ \imprimirorientador}
		\end{SingleSpacing}
		\vspace*{\fill}
		\vspace*{\fill}

	% g)local (cidade) da instituição onde vai ser apresentado o
	% trabalho. No caso de cidades homônimas, recomenda-se o
	% acréscimo da sigla da unidade da federação;
		\MakeTextUppercase{\imprimirlocal}

	% h)ano de entrega, em algarismos arábicos
		\MakeTextUppercase{\imprimirdata}
		\vspace*{1cm}
	\end{center}
	\clearpage

	% verso
	% devem constar os dados internacionais
	% de catalogação-na-publicação (ficha catalográfica), conforme o Código
	% de Catalogação Anglo-Americano (AACR2), que devem ser elaborados
	% pela biblioteca que atende ao curso em que o trabalho foi apresentado
	\begin{fichacatalografica}

	% ficha catalográfica é impressa obrigatoriamente no verso da folha de rosto [1.2.1 d]
	\ifthenelse{\NOT\boolean{@twoside}}{\setcounter{page}{\value{page}-1}}
	\rmfamily
		\vspace*{15cm}
		% Posição vertical
		\hrule
		% Linha horizontal
		\begin{center}
	% Minipage Centralizado
			\begin{minipage}[c]{12.5cm} % Largura
				\imprimirautor
				\hspace{0.5cm} \imprimirtitulo / \imprimirautor. --
				\imprimirlocal, \imprimirdata-
				\hspace{0.5cm} \pageref{LastPage} p. : il.; 30 cm.\\
				\hspace{0.5cm} \imprimirorientadorRotulo \imprimirorientador\\
				\hspace{0.5cm} \imprimircoorientadorRotulo \imprimircoorientador\\
				\hspace{0.5cm}
				\parbox[t]{\textwidth}{\imprimirtipotrabalho~--~\imprimirinstituicao,
				\imprimirdata.}\\
				\hspace{0.5cm}
				1. Goertzel.
				2. FPGA.
				I. Professor Dr. Jarbas Aryel Nunes da Silveira.
				II. Universidade Federal do Ceará.
				III. Centro e Tecnologia.
				IV. Um core de processamento digital de sinais monotônicos para telefonia\\
				%\hspace{8.75cm} CDU 02:141:005.7\\
			\end{minipage}
		\end{center}
		\hrule
	\end{fichacatalografica}
	\cleardoublepage
	
\end{document}
