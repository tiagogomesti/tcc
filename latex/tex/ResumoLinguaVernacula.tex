\documentclass[monografia]{subfiles}

\begin{document}
	% Inicia-se em folha/página distinta com a palavra RESUMO
	\begin{resumo}
	\noindent

	Com a tecnologia tendo grandes avanços nas últimas décadas, a taxa de renovação de dispositivos eletrônicos aumenta também, fazendo com
	que alguns dispositivos tenham sua fabricação descontinuada. Este trabalho tem como um de seus propósitos evitar um novo reprojeto das funções
	de processamento de sinalização monotônica, pois neste trabalho foi implementado um core de processamento descrito em HDL, sendo portável para
	diferentes FPGAs, pois a linguagem de descrição de hardware é universal.
	O core descrito neste trabalho é um módulo controlado, a ser inserido em um sistema, que tem como propósito ser o bloco responsável pelo processamento
	matemático dos sinais monotônicos do sistema.
	No desenvolvimento deste trabalho foi aplicado o algoritmo de Goertzel, que é derivado da
	transformada discreta de Fourier. Os resultados dos tempos de processamentos mostram que o core proposto neste trabalho tem uma bom desempenho, alcançando 
	um processamento máximo de 222 canais, tendo como meta inicial 8, em um intervalo de tempo de $125 \mu s$, 
	denotando um alto poder de processamento para um valor de clock de apenas
	$80 MHz$. 






	% Apresentação concisa dos pontos
	% relevantes do documento, fornecendo uma visão rápida e clara do
	% conteúdo e das conclusões do trabalho. Elaborado de acordo com a NBR
	% 6028/2003, conforme as seguintes orientações: 

	% a) o resumo deve ser informativo, apresentando finalidades,	metodologia, resultados e conclusões;
	
	% b) composto de uma sequência de frases concisas, afirmativas e não de enumeração de tópicos;

	% c) deve-se usar parágrafo único e justificado;
	
	% d) usar o verbo na voz ativa e na 3a pessoa do singular;
	
	% e) o resumo expresso em trabalhos acadêmicos (teses, dissertações e outros) deve conter de 150 a 500 palavras;
	
	% f) a primeira frase do resumo deve ser significativa e expressar o tema principal do trabalho;
	
	% g) deve ser evitado o uso de frases negativas, símbolos e
	
	% fórmulas que não sejam de uso corrente, comentário pessoal,
	% críticas ou julgamento de valor.

	% h)as palavras-chave devem figurar logo abaixo do resumo,
	% antecedidas da expressão “Palavras-chave:” separadas e
	% finalizadas por ponto
	\vspace{\onelineskip}
	\textbf{Palavras-chave}: Goertzel. FPGA. DSP. VHDL.
	\end{resumo}
		
\end{document}
